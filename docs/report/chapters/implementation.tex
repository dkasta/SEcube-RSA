\chapter{Design and implementation}
\section{Hardware and software partitioning}
Before working on the actual implementation it is necessary to decide what
functionality should be implemented in hardware (on the FPGA) and in software
(running on the CPU).

The most efficient solution in terms of performance and power consumption is to
design at RTL the complete RSA module (composed of a key generation module and
an encryption/decryption module), but the available FPGA provides 7000 LUTs
only, therefore it is unfeasible to fit the entire design in HW.

In order to reduce the FPGA area occupation of the design, the natural solution
was to identify the most critical parts, to be implemented in HW, and implement
the rest in SW.

Since keys are generated once and used many times, the key generation module is
less critical and it could be implemented in SW.  The encryption/decryption
module is basically a modular exponential: its critical part is the modular
multiplier.

In this scenario it was necessary to find a reference architecture of a modular
multiplier designed to be area efficient: the most widely accepted solution is
the Montgomery multiplier, but all the implementations found in literature
require a too high number of LUTs (e.g. for a 1024 bits multiplier, which is
the bare minimum key length for a basic level of security: Blum's \cite{blum}
smallest implementation requires 7572 LUTs, Zhang's \cite{ZHANG2007456} 9024
LUTs, Amanor's \cite{amanor} 7680 LUTs, Daly's
\cite{Daly02efficientarchitectures} 10806 LUTs).

The Tenca and Koc's \cite{tenca_koc} architecture is theoretically adaptable to
any area at disposal: the more area is available, the higher performance is
achievable.

Since the available area in the SEcube FPGA is very low and the time overhead
for transferring data between CPU and FPGA would be non-negligible (in the
smallest possible case, with 1024 bits keys, for a single multiplication it is
needed to send 2048 bits for the multiplicands, 1024 bits for the modulus and
receive 1024 bits of product, over a 16 bits bus), the performance gain would
be low or null.

Moreover, since RSA typical applications are to encrypt symmetric keys or
message digests, performance are not very critical: what it is really critical
is the implementation security, which is easier to obtain in software, thanks
to the easier verification process.

After all these considerations, the best solution appeared to be the usage of a
well tested and trusted software library for the implementation of RSA
functionality.

\section{Firmware side}
The SEcube firmware has been extended in order to support receiving, processing
and responding to RSA functionality and key management and X.509 certificate
management requests sent from the host.

\subsection{RSA and X.509 library}
The most convenient solution for getting core RSA functionalities (encrypt,
decrypt, sign, verify) and X.509 certificates generation is to exploit an
existing library.
Since the library should run on an STM32F429 micro controller, it should have a
small resources footprint and it should be implemented in C, being the only
natively supported programming language.
The library should also be included into the SEcube open source program,
therefore it must have a permissive license.
The most suitable libraries complying with the desirable characteristics are:
\texttt{mbedtls} \cite{mbedtls}, \texttt{libtomcrypt} \cite{libtomcrypt} and
\texttt{BearSSL} \cite{bearssl}.
Considering stability, security and trustfulness of the developer,
\texttt{mbedtls} by \texttt{ARM} has been chosen as the underlying library for
this project.

\bigskip
The library has been configured to run on the SEcube processor and all the
unused files have been removed.

The SEcube True Random Number Generator is used whenever a random number has to
be generated.

\bigskip
The digest for the digital signature is computed by the SHA256 algorithm.

\subsection{Flash memory}
\subsubsection{RSA key storage}
\begin{itemize}
	\item \textbf{Key structure}
		\begin{lstlisting}
typedef struct se3_rsa_flash_key_ {
	uint32_t id;
	uint16_t key_size;
	uint8_t type;
	uint8_t public_only;
	uint8_t* N;
	uint8_t* E;
	uint8_t* D;
} se3_rsa_flash_key;
		\end{lstlisting}
		\texttt{se3\_rsa\_flash\_key} is the structure holding all the
		information needed for storing RSA keys to flash.
		In particular it contains:
		\begin{itemize}
			\item \texttt{id}: the unique identifier of the key.
			\item \texttt{key\_size}: the size of \texttt{N},
				\texttt{E} and \texttt{D} arrays.
			\item \texttt{type}: it specifies the operations that
				can be performed with the key (crypto only,
				signature only or both)
				(see~\ref{subsubsec:perf_and_sec} for more
				details).
			\item \texttt{public\_only}: it is set to \texttt{true}
				when the key is composed of its public part
				only (therefore \texttt{D} content is ignored).
			\item \texttt{N}, \texttt{E} and \texttt{D}: the raw key
				data.
		\end{itemize}

	\item \textbf{Flash memory} \\
		In order to avoid code duplication for better maintainability
		and integration, RSA keys are stored to flash memory in the same
		flash node type (\texttt{se3\_flash\_key}) and in the same IDs
		space as symmetric keys, using the functions of
		\texttt{se3\_key}.

		\bigskip
		\texttt{se3\_rsa\_keys.c} includes all the functions required
		for managing RSA keys stored to flash.
		It operates as an interface for RSA keys over the functions
		provided by \texttt{se3\_keys.c}: \texttt{se3\_flash\_key}
		contains a single data buffer, while
		\texttt{se3\_rsa\_flash\_key} contains multiple data fields,
		therefore it is necessary to concatenate the different slices
		into a single buffer when writing to flash
		(performed by the \texttt{se3\_rsa\_to\_plain\_flash} function)
		and to split into the different fields when reading from flash
		(performed by the \texttt{se3\_plain\_to\_rsa\_flash} function).
\end{itemize}

\subsubsection{X.509 certificates storage}
The SEcube SDK did not provide any kind of certificate support, therefore a new
flash node type have been implemented specifically for X.509 certificates,
together with the functions needed for reading and writing to and from flash, in
\texttt{se3\_x509.c}.

\subsection{Dispatcher}
When the host sends a request to the SEcube device, the request is processed by
the \emph{Communication core} and forwarded to the \emph{Dispatcher core} which
is in charge of calling the requested functionality.  Since the
\emph{Dispatcher core} can fit a limited number of functionalities, RSA
functionalities are not directly called: \emph{Dispatcher core} calls a
dedicated \emph{RSA dispatcher core}, which in turn calls the specific RSA
functionality.

\subsubsection{\texttt{se3\_rsa}}
\texttt{se3\_rsa.c} exposes all the RSA functionalities to the dispatcher.
All the \texttt{se3\_rsa} functions have a similar structure:
\begin{enumerate}
	\item Parse the request from the host, checking its validity.
	\item Perform the RSA computation by exploiting the underlying RSA
		library.
	\item Build the response to the host.
\end{enumerate}

Among the request validity checks there is the key type check: if the request
asks to perform a crypto operation using a signature-only key or a signature
operation using a crypto-only key, an error code is returned, instead of
performing the requested operation.

\section{Host side}
The SEcube host side code has been extended in order to add APIs for accessing
RSA functionality and key management and X.509 certificate management of the
SEcube device.
These APIs also take care of checking correctness of inputs and outputs.

This extension has been performed while trying to maximize the integration with
the existing APIs: some APIs have been reused (\texttt{L1FindKey}), some have
been extended to support additional features (\texttt{L1KeyEdit},
\texttt{L1Encrypt}, \texttt{L1Decrypt}), while APIs peculiar to RSA and X.509
certificates have been added trying to comply with existing hierarchies and
conventions.

\subsection{Communication timeout}
RSA key generation is a very complex operation, since it has to generate very
long random numbers (at least 1024 bits) until it generates a prime number,
therefore it may take a huge amount of time.

The communication between SEcube device and host was limited by a timeout of 10
seconds (defined in \texttt{L0.h} as \texttt{SE3\_TIMEOUT}), resulting in an
exception before RSA key generation could complete.

\bigskip
According to empirical measurements, 1000 seconds should be enough for
generating 2048 bits keys, therefore the timeout has been set to that value.

It is advisable to use longer timeouts when generating longer RSA keys.

