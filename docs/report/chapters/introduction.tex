\chapter{Introduction}
This manual covers the details and the features of the Asymmetric cryptography
implementation of the SEcube\textsuperscript{\texttrademark} Open Source
Security Platform. To learn more about the
SEcube\textsuperscript{\texttrademark} and the capabilities of this Open Source
Security Platform, please check out the SEcube™ SDK documentation.  Asymmetric
cryptography implementation concerns asymmetric key management, an RSA cipher,
Digital signature and X.509 certificates.

\section{Cryptography}
Cryptography is the discipline in charge of developing techniques for securing
communication in presence of opponents willing to read and alter messages.
Encryption uses an algorithm and a key to transform an input (i.e., plaintext)
into an encrypted output (i.e., ciphertext).

Algorithms are considered secure if an attacker cannot determine any properties
of the plaintext or key, given the ciphertext. An attacker should not be able
to determine anything about a key given a large number of plaintext/ciphertext
combinations which used the key.

There are typically two main kind of cryptographic solutions:
\begin{itemize}
	\item \emph{Symmetric cryptography}: the actors involved in the
		communication must share a common secret to secure the
		communication. Performance is generally acceptable, therefore
		it is suitable for encrypting large amounts of data.
		Communication problem is not completely solved: the common
		secret still has to be confidentially shared.
	\item \emph{Asymmetric cryptography}: the actors can securely
		communicate without any shared common secret.
		Communication problem is solved, but it is necessary to pay in
		terms of performance.
\end{itemize}

\subsection{Asymmetric cryptography}
\subsubsection{Characteristics}
In an asymmetric (or public-key) cryptosystem each actor owns a pair of keys:
private, known to the owner only, and public, known to anyone.
There is a single algorithm $F(key, msg)$ which can be applied to different
keys, thanks to its reciprocal properties:
\begin{itemize}
	\item \emph{Encryption:} $ciphertext = F(key_{pub, recipient}, plaintext)$
	\item \emph{Decryption:} $plaintext = F(key_{priv, recipient}, ciphertext)$
\end{itemize}

In order to counteract brute force attacks, the keys must be very long (i.e.
minimum 2048 bits for an acceptable security level, with today algorithms and
technology).

\subsubsection{Applications}
\begin{itemize}
	\item \textbf{Symmetric key distribution}
		To solve the secure communication problem between actors $A$ and $B$:
		\begin{enumerate}
			\item $A$ generates a symmetric cryptosystem secret.
			\item $A$ encrypts the secret using $B$'s public key.
			\item $A$ sends the encrypted secret over an untrusted channel to $B$.
			\item $B$ decrypts the secret using its own private key.
			\item $A$ and $B$ now share the common secret and can start using symmetric
				cryptography.
		\end{enumerate}

		This solution guarantees confidentiality.

	\item \textbf{Digital signature}
		\begin{itemize}
			\item \emph{Sign:} the signer computes the digest (e.g. SHA256) of the
				message to be signed, then applies the asymmetric cryptography
				algorithm using its private key.
			\item \emph{Verify:} the verifier computes the digest of the signed message
				and applies the asymmetric cryptography algorithm using the signer
				public key: if the two digests match, the signature is
				verified.
		\end{itemize}
		This solution guarantees authentication and integrity, moreover, if keys are
		taken from certificates by Certification Authorities,
		non-repudiation is guaranteed too.
\end{itemize}

\subsection{RSA cryptography}
The RSA algorithm is the basis of a public-key cryptosystem, a suite of
cryptographic algorithms used for security purposes.
RSA was first described publicly in 1977 by Ron Rivest, Adi Shamir and Leonard
Adleman of the Massachusetts Institute of Technology, although the 1973
creation of a public key algorithm by British mathematician Clifford Cocks has
been kept secret by the UK's GCHQ until 1997.
PKCS (Public Key Cryptography Standards) are specifications used in computer
cryptography. PKCS\#1 \cite{rsa}, in particular, identifies the RSA
cryptosystem standard.
\subsubsection{Algorithm}
\begin{itemize}
	\item \textbf{Key generation}:\\
		\begin{enumerate}
			\item select $P$ and $Q$ such that they are prime, big, random and secret.
			\item let $N = P \cdot Q$ and $\varphi = (P - 1) \cdot (Q - 1)$
			\item select $E$ such that $1 < E < \varphi$ and $E$ and $\varphi$ are coprime
			\item let $D = E^{-1} \mod \varphi$
			\item public key is composed of $\{E, N\}$ and can be shared to anyone;
				private key is composed of $\{D, N\}$ and must be kept secret
		\end{enumerate}
	\item \textbf{Modular exponential}:\\
		the keys are chosen in such a way that: $(P^E)^D \mod N \equiv (P^D)^E \mod
		N$, therefore the primitives provided by PKCS\#1 are:
		\begin{itemize}
			\item \emph{RSA Encryption Primitive}:\\
				anyone (since $E$ and $N$ are public) can encrypt a message by applying:
				\[ciphertext = (plaintext)^E \mod N\]
			\item \emph{RSA Decryption Primitive}:\\
				the owner of the key only (since $D$ is private) can decrypt the
				ciphertext by applying:
				\[plaintext = (ciphertext)^D \mod N\]
			\item \emph{RSA Signature Primitive}:\\
				the owner of the key only (since $D$ is private) can sign a message by
				applying:
				\[signature = (hash(message))^D \mod N\]
			\item \emph{RSA Verification Primitive}:\\
				anyone (since $E$ and $N$ are public) can verify a signature by:
				\begin{itemize}
					\item retrieving the signed has by applying:
						\[hash = (signature)^E \mod N\]
					\item computing the $hash$ of the $message$ again
					\item comparing the two hashes
				\end{itemize}
		\end{itemize}
\end{itemize}

\subsubsection{Performance and security}\label{subsubsec:perf_and_sec}
\begin{itemize}
	\item \textbf{Key length}:\\
		The factorization complexity is $O(e^n)$, therefore, in order
		to make it unfeasible to break RSA by means of brute force, it
		is necessary to use very long keys (1024 bits for a bare
		minimum level of security, but 2048 bits or more are
		recommended).

		The exponentiation complexity is instead $O(\log^3n)$,
		therefore RSA performance are low and it is discouraged to
		encrypt long messages with such an algorithm, but it is
		suitable for secure key exchange and digital signature.

	\item \textbf{Public key optimizations}:\\
		Execution time of exponential highly depends on the number of
		\texttt{1}s in the exponent (in binary format), therefore $E$
		is usually chosen between 3, 17 or 65537, since they are prime
		and their binary representations contain 2 \texttt{1}s only.
		65537 is preferable, since large exponents guarantees higher
		security levels.

	\item \textbf{Private key optimizations}:\\
		The Chinese Remainder Theorem can be used to make private-key
		operations up to 4 times faster.

	\item \textbf{Dedicated keys}:\\
		When providing RSA functionalities as a service (e.g., from a
		server which accepts messages and encrypts/signs them) it is
		important to use separate keys for encryption and for signing.
		Otherwise an attacker could compute the hash of a message, ask
		to decrypt it and then claim it was actually signed with that
		particular key.
\end{itemize}

\subsection{X.509 certificates}
X.509 is a standard format for public key certificates, digital documents that
securely associate cryptographic key pairs with identities such as websites,
individuals, or organizations.  It is based on the internationally trusted
International Telecommunications Union (ITU) X.509 standard, which defines the
format of public key infrastructure (PKI) certificates. They are used to manage
identity and security in internet communications and computer networking. They
are unobtrusive and ubiquitous, and we encounter them every day when using
websites, mobile apps, online documents, and connected devices.

\section{SEcube\textsuperscript{\texttrademark}}
SEcube\textsuperscript{\texttrademark} is an open-source security-oriented
hardware and software platform.

\subsection{Hardware}
The core of the SEcube\textsuperscript{\texttrademark} Hardware device family
is a chip which embeds three hardware components: an STM32F429 processor (which
includes an ARM Cortex M4 core and a True Random Number Generator), a Lattice
MachXO2-7000 FPGA and an Infineon SLJ52G EAL5+ certified smart-card.\\ The
Lattice MachXO2-7000 device is based on a fast, non-volatile logic array whose
main features are: 7,000 LUTs, 240 Kbits of embedded block RAM, 256 Kbits of
user Flash memory.  The communication between FPGA and CPU can be performed
through a 16-bit internal bus.

\subsection{Software}
SEcube\textsuperscript{\texttrademark} SDK is composed of:
\begin{itemize}
	\item \emph{Host}: code to be run on a desktop computer which provides
		libraries to control SEcube and expose its functionalities.
	\item \emph{Firmware}: code to be run on a SEcube device which provide
		libraries for communicating with host and the implementations of
		symmetric cryptosystems and hashing functions.
\end{itemize}
The purpose of this project is to extend the SDK introducing the basics of
asymmetric cryptography.

