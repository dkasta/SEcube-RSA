\chapter{API documentation}
This section contains an overview of the level 1 APIs which expose asymmetric
cryptography functionalities. More details can be found in the source code
Doxygen-compatible comments.

\section{SEcube RSA cryptography}
\paragraph{Key management} \hspace{0pt} \\
\begin{lstlisting}
typedef struct se3Key_ {
	uint32_t id;
	uint16_t dataSize;
	uint8_t* data;
	se3AsymmKey asymmKey;
} se3Key;

typedef struct se3AsymmKey_ {
	uint8_t* N;
	uint8_t* E;
	uint8_t* D;
	uint8_t type;
} se3AsymmKey;

struct RSAKeyType {
	enum {
		SE3_RSA_KEY_GENERIC,
		SE3_RSA_KEY_CIPHER,
		SE3_RSA_KEY_SIGN
	};
};
\end{lstlisting}

\texttt{se3Key} is the structure containing symmetric or asymmetric keys.
\texttt{id} is a 32-bit unique identifier, \texttt{datasize} is the key size,
\texttt{data} is the pointer to the symmetric key value and \texttt{asymmKey}
is a structure containing the asymmetric key data.
The asymmetric key may be of signature type, encryption type or a generic type.

\begin{lstlisting}
struct KeyOpEdit {
	enum {
	  SE3_KEY_OP_ADD = 1,
	  SE3_KEY_OP_DELETE = 2,
	  SE3_KEY_OP_ADD_TRNG = 3,
	  SE3_KEY_OP_ADD_RSA = 5,
	  SE3_KEY_OP_ADD_GEN_RSA = 6
	};
};

void L1::L1KeyEdit(se3Key& k, uint16_t op);
\end{lstlisting}

\texttt{L1KeyEdit} provides write access to RSA keys on SEcube device.
It can perform three operations, depending on \texttt{op} value:
\begin{itemize}
	\item \texttt{SE3\_KEY\_OP\_ADD\_GEN\_RSA}: generate an RSA key \texttt{k.dataSize} Bytes long and store
	    it to SEcube flash memory, associating it with the ID specified in
		\texttt{k.id}.
    \item \texttt{SE3\_KEY\_OP\_ADD\_RSA}: store the key provided in
	    \texttt{k.asymmKey} to SEcube flash memory, associating it with the
		ID specified in \texttt{k.id}.
    \item \texttt{SE3\_KEY\_OP\_DELETE}: delete the key associated with the ID
	    specified in \texttt{k.id} from SEcube flash memory (can be a
		symmetric or an asymmetric key).
\end{itemize}

Note 1: Adding RSA key of size different from 1024, 2048, 4096 and 8192 bits is
forbidden and generates an error.

Note 2: \texttt{SE3\_KEY\_OP\_ADD} and \texttt{SE3\_Key\_OP\_ADD\_TRNG} are
related to symmetric key functionalities.

\begin{lstlisting}
void L1::L1FindKey(uint32_t keyId, bool& found);
\end{lstlisting}

\texttt{L1FindKey} sets \texttt{found} to \texttt{true} if a key (symmetric or
asymmetric) associated with \texttt{keyId} is stored to SEcube flash memory, to
\texttt{false} otherwise.

\begin{lstlisting}
void L1::L1AsymmKeyGet(se3Key& k);
\end{lstlisting}

\texttt{L1AsymmKeyGet} reads from SEcube flash memory the public part (modulo
$N$ and public exponent $E$) associated with the ID specified in \texttt{k.id}
and stores it to \texttt{k}.

\bigskip
Note: for security reasons there is no API for reading private exponent $D$ of
an RSA key stored to SEcube flash memory.

\paragraph{Cryptography} \hspace{0pt} \\
\begin{lstlisting}
void L1::L1Encrypt(size_t plaintext_size, std::shared_ptr<uint8_t[]> plaintext,
        SEcube_ciphertext& encrypted_data, uint16_t algorithm,
        uint16_t algorithm_mode,
        se3Key key, uint8_t on_the_fly);
\end{lstlisting}

When \texttt{algorithm} is set to \texttt{L1Algorithms::Algorithms::RSA} the
data stored to \texttt{plaintext} is encrypted using the RSA algorithm. If
\texttt{on\_the\_fly} is set to \texttt{true} the key stored to \texttt{key} is
used, while if it is set to \texttt{false} the key stored to SEcube flash
memory associated with the ID \texttt{key.id} is used.

The generated ciphertext and other data related to the encryption operation are
outputted to \texttt{encrypted\_data}.

\bigskip
\begin{itemize}
	\item Important note: \texttt{plaintext\_size} can be at most
		\texttt{key\_size - 2 - 2 * 32} bytes long, where
		\texttt{key\_size} is the size of the key to be used in the
		encryption operation and \texttt{32} is the size of the output
		of the SHA256 hash function used in the encryption operation.
		Greater \texttt{plaintext\_size} would raise an error.

		If the user needs to encrypt longer messages he is in charge of
		splitting messages into shorter chunks.
		This is discouraged due to performance and security issues
		\cite{rsa_long_msg}: consider using RSA for securely distribute
		a symmetric key to be used for encrypting the long message.

	\item Note: \texttt{algorithm\_mode} is ignored when \texttt{L1Encrypt}
		uses the RSA algorithm.
\end{itemize}

\begin{lstlisting}
class SEcube_ciphertext{
public:
	se3Key key;
	uint16_t algorithm;
	std::unique_ptr<uint8_t[]> ciphertext;
	size_t ciphertext_size;
	...
};

void L1::L1Decrypt(SEcube_ciphertext& encrypted_data, size_t& plaintext_size,
        std::shared_ptr<uint8_t[]>& plaintext, uint8_t on_the_fly);
\end{lstlisting}

When \texttt{encrypted\_data.algorithm} is set to
\texttt{L1Algorithms::Algorithms::RSA} the data stored to
\texttt{encrypted\_data.ciphertext} is decrypted using the RSA algorithm.
If \texttt{on\_the\_fly} is set to \texttt{true} the key stored to
\texttt{encrypted\_data.key} is used, while if it is set to \texttt{false} the
key stored to SEcube flash memory associated with the ID
\texttt{encrypted\_data.key.id} is used.

The generated plaintext and its length will be stored to \texttt{plaintext} and
\texttt{plaintext\_size}.

\paragraph{Digital signature} \hspace{0pt} \\
\begin{lstlisting}
void L1::L1Sign(const size_t input_size, const std::shared_ptr<uint8_t[]> input_data,
		const se3Key key, uint8_t on_the_fly, size_t &sign_size,
		std::shared_ptr<uint8_t[]> &sign);
\end{lstlisting}

\texttt{L1Sign} computes the RSA signature of \texttt{input\_size} bytes of
data stored to \texttt{input\_data}.
The result is stored to \texttt{sign} and its size to \texttt{sign\_size}.
If \texttt{on\_the\_fly} is set to \texttt{true} the key stored to \texttt{key}
is used, while if it is set to \texttt{false} the key stored to SEcube flash
memory associated with the ID \texttt{key.id} is used.

\begin{lstlisting}
void L1::L1Verify(const size_t input_size,
		const std::shared_ptr<uint8_t[]> input_data,
		const se3Key key, uint8_t on_the_fly, const size_t sign_size,
		const std::shared_ptr<uint8_t[]> sign, bool &verified)
\end{lstlisting}
\texttt{L1Verify} checks the validity of the RSA signature stored to
\texttt{sign} and \texttt{sign\_size} bytes long, associated to the message
stored to \texttt{input\_data} of \texttt{input\_size} bytes.
If \texttt{on\_the\_fly} is set to \texttt{true} the key stored to \texttt{key}
is used, while if it is set to \texttt{false} the key stored to SEcube flash
memory associated with the ID \texttt{key.id} is used.
If the signature is valid \texttt{verified} is set to \texttt{true}, to
\texttt{false} otherwise.

\subsection{Use cases}
This section contains some code examples of the most common use cases of the
RSA library.

See \texttt{src/examples} for the complete source code.

\subsubsection{Key generation}
The following code snippet shows how to generate an RSA key \texttt{KEY\_SIZE}
long (in Bytes) of type \texttt{KEY\_TYPE} and store it to flash, associating
it to \texttt{KEY\_ID} id.
\begin{lstlisting}
unique_ptr<L1> l1 = make_unique<L1>();

// login to the SEcube device
...

// specify the key ID, size and type
se3Key key = {
	.id = KEY_ID,
	.dataSize = KEY_SIZE,		// (in Bytes) can be 128, 256, 512 or 1024
	.asymmKey = {
		.type = KEY_TYPE	// can be any value of L1Key::RSAKeyType
	}
};

// send the request for generating the key and wait for the response
l1->L1KeyEdit(key, L1Commands::KeyOpEdit::SE3_KEY_OP_ADD_GEN_RSA);
\end{lstlisting}
\subsubsection{Symmetric key distribution}
The following code snippets show how to securely distribute a secret key from
the \emph{Sender} to the \emph{Receiver}, according to these steps:
\begin{enumerate}
	\item \emph{Receiver} reads its public key from flash and sends it to
		\emph{Sender}.
	\item \emph{Sender} encrypts the secret key using the \emph{Receiver}
		public key and sends it to \emph{Receiver}.
	\item \emph{Receiver} decrypts the secret key using its private key.
\end{enumerate}

\bigskip
\begin{itemize}
	\item \emph{Sender} side:
\begin{lstlisting}
unique_ptr<L1> l1 = make_unique<L1>();

// login to the SEcube device
...

// receive the RSA key to use for encrypting data from Receiver
se3Key asymmKey;
...

SEcube_ciphertext cipher;

// send the request for encrypting the symmetric key and wait for the response
// (assuming SYMM_KEY is a buffer containing the symmetric key sized
// SIMM_KEY_SIZE)
l1->L1Encrypt(SYMM_KEY_SIZE, SYMM_KEY, cipher, L1Algorithms::Algorithms::RSA,
		0, asymmKey, true);

// send cipher to Receiver
...
\end{lstlisting}
	\item \emph{Receiver} side:
\begin{lstlisting}
unique_ptr<L1> l1 = make_unique<L1>();

// login to the SEcube device
...

// specify the ID of the RSA key to use for encrypting data
se3Key asymmKey = {
	.id = ASYMM_KEY_ID
};

// send the request for getting the public part of the key to be used for
// encrypting the symmetric key and wait for the response
// (assuming an RSA key associated with ASYMM_KEY_ID is stored
// to the SEcube flash memory)
l1->L1AsymmKeyGet(asymmKey);

// send asymmKey to Sender
...

// receive cipher from Sender
SEcube_ciphertext cipher;
...

shared_ptr<uint8_t[]> symmKey;
size_t symmKeySize;

cipher.key.id = ASYMM_KEY_ID;

// send the request for decrypting the symmetric key and wait for the response
l1->L1Decrypt(cipher, symmKeySize, symmKey, false);
\end{lstlisting}
\end{itemize}
\subsubsection{Digital signature}
The following code snippets show how the \emph{Signer} signs a message and the
\emph{Verifier} verifies it, according to these steps:
\begin{enumerate}
	\item \emph{Signer} reads its public key from flash and sends it to
		a trusted key storage (such as a key server).
	\item \emph{Signer} signs the message with its private key and sends
		the original message and its signature to the \emph{Verifier}.
	\item \emph{Verifier} receives the message and its signature from the
		\emph{Signer} and the \emph{Signer} public key from the trusted
		key storage.
	\item \emph{Verifier} verifies the message.
\end{enumerate}

\bigskip
\begin{itemize}
	\item \emph{Signer} side:
\begin{lstlisting}
unique_ptr<L1> l1 = make_unique<L1>();

// login to the SEcube device
...

// specify the ID of the RSA key to use for signing data
// (assuming an RSA key associated with KEY_ID is stored
// to the SEcube flash memory)
se3Key key = {
	.id = KEY_ID
};

// send the request for getting the public part of the key to be used for
// verifying the message and wait for the response
l1->L1AsymmKeyGet(key);

// send key to the trusted key storage
...

shared_ptr<uint8_t[]> sign;
size_t signature_size;

// send the request for signing the message and wait for the response
// (assuming MESSAGE is a buffer containing the message sized
// MESSAGE_SIZE)
l1->L1Sign(MESSAGE_SIZE, MESSAGE, key, false, signature_size, signature);

// send MESSAGE, MESSAGE_SIZE, signature and signature_size to Verifier
...
\end{lstlisting}
	\item \emph{Verifier} side
\begin{lstlisting}
unique_ptr<L1> l1 = make_unique<L1>();

// login to the SEcube device
...

// receive message, message_size, signature and signature_size from Signer
// and key from trusted storage
shared_ptr<uint8_t[]> message;
size_t message_size;
shared_ptr<uint8_t[]> signature;
size_t signature_size;
se3Key key;
...

// send the request for verifying the signature and wait for the response
bool verified;
l1->L1Verify(message_size, message, key, true,
		signature_size, signature, verified);
\end{lstlisting}
\end{itemize}

\section{SEcube X.509 certificates}
The functions manage the certificates stored in the SEcube device. They build
the request buffer according to the desired operation and they check and parse
the response buffer after the SEcube computations. They throw exceptions in
case of errors.

\begin{lstlisting}
struct SEcube_certificate_info{
	uint32_t cert_id;
	uint32_t issuer_key_id;
	uint32_t subject_key_id;
	std::string serial_str;
	std::string not_before;
	std::string not_after;
	std::string issuer_name;
	std::string subject_name;
};
\end{lstlisting}

\texttt{SEcube\_certificate\_info} is the data structure containing all the
information required for the generation of an X.509 certificate.
In particular, the fields are:
\begin{itemize}
	\item \texttt{cert\_id} is the ID to associate the certificate with,
		when storing it to SEcube flash memory.
	\item \texttt{issuer\_key\_id} and \texttt{subject\_key\_id}
		are the IDs of the issuer and subject keys stored to SEcube
		flash memory.
	\item \texttt{serial\_str} is a string containing the serial number of
		the certificate (in hexadecimal format).
	\item \texttt{not\_before} and \texttt{not\_after} are strings
		containing the start and the end of the validity period
		(in \texttt{YYYYMMDDhhmmss} format).
	\item \texttt{issuer\_name} and \texttt{subject\_name} are strings
		containing comma-separated list of OID types and values
		(e.g. \texttt{"C=UK,O=ARM,CN=mbed TLS CA"}) related to issuer
		and subject.
\end{itemize}

\begin{lstlisting}
enum CertOpEdit {
	SE3_CERT_OP_ADD = 0,
	SE3_CERT_OP_DELETE = 1
};
	
void L1::L1CertificateEdit(const L1Commands::CertOpEdit op,
		const SEcube_certificate_info info);
\end{lstlisting}
\texttt{L1CertificateEdit} is the function for managing certificates, the
operation to be executed has to be specified in the first parameter. The
certificate can be added or deleted.
The second parameter is the certificate.
\texttt{L1KeyEdit} provides write access to X.509 certificates on SEcube device.
It can perform three operations, depending on \texttt{op} value:
\begin{itemize}
	\item \texttt{SE3\_CERT\_OP\_ADD}: generate an X.509 certificate according
		to the information contained in \texttt{info} and store it to SEcube
		flash memory, associating it with the ID specified in
		\texttt{info.cert\_id}.
	\item \texttt{SE3\_CERT\_OP\_DELETE}: delete the certificate associated
		with the ID specified in \texttt{info.cert\_id} from SEcube flash
		memory.
\end{itemize}

\begin{lstlisting}
void L1::L1CertificateGet(const uint32_t cert_id, std::string &cert);
\end{lstlisting}
\texttt{L1CertificateGet} reads from SEcube flash memory the X.509 certificate
associated with the ID specified in \texttt{cert\_id} and stores it to
\texttt{cert}, in \texttt{PEM} format.

\begin{lstlisting}
void L1::L1CertificateFind(uint32_t certId, bool& found);
\end{lstlisting}
\texttt{L1CertificateFind} sets \texttt{found} to \texttt{true} if an X.509
certificate associated with \texttt{certId} is stored to SEcube flash memory,
to \texttt{false} otherwise.

\begin{lstlisting}
void L1::L1CertificateList(std::vector<uint32_t>& certList);
\end{lstlisting}
\texttt{L1CertificateList} returns the list of IDs of all the certificates
stored to the SEcube device flash memory.

\subsection{Use cases}
This section contains some code examples of the most common use cases of the
X.509 certificates library.

See \texttt{src/examples} for the complete source code.

\subsubsection{Certificate generation}
The following code snippet shows how to generate an X.509 certificate and store
it to flash, associating it with \texttt{CERT\_ID} id.
\begin{lstlisting}
unique_ptr<L1> l1 = make_unique<L1>();

// login to the SEcube device
...

// specify the certificate information
SEcube_certificate_info info = {
	.cert_id = CERT_ID,
	.issuer_key_id = ISSUER_KEY_ID,
	.subject_key_id = SUBJECT_KEY_ID,
	.serial_str = SERIAL,
	.not_before = NOT_BEFORE,
	.not_after = NOT_AFTER,
	.issuer_name = ISSUER_NAME,
	.subject_name = SUBJECT_NAME
};

// send the request for generating the key and wait for the response
l1->L1CertificateEdit(L1Commands::CertOpEdit::SE3_CERT_OP_ADD, info);
\end{lstlisting}

